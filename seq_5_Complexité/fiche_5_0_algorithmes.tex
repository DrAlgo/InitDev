\documentclass[11pt,a4paper]{article}


\usepackage{amssymb,amsmath,amsfonts}    %ams
\usepackage{wasysym} %des symboles
%\usepackage{a4wide}
\usepackage[tmargin=1in,bmargin=1in,lmargin=.75in,rmargin=.75in]{geometry}
\usepackage{graphicx}
%\usepackage{pstricks}
%\usepackage{multido}
\usepackage{verbatim}
\usepackage{enumerate}
\usepackage{tikz}
\usetikzlibrary{calc,positioning,backgrounds}

\usepackage[utf8]{inputenc} 
%\usepackage[T1]{fontenc}
\usepackage{listings}

\newcommand{\R}{{\mathbb R}}   % reals
\newcommand{\Q}{{\mathbb Q}}   % rationals
\newcommand{\N}{{\mathbb N}}   %natural numbers
\newcommand{\Z}{{\mathbb Z}}    %integers
\renewcommand{\P}{{\mathbb P}}   %primes
\newcommand{\F}{{\mathbb F}}

\newcommand\cc{{\cal C}}
\newcommand{\cw}{{\cal W}}



\newtheorem{theorem}{Theorem}
\newtheorem{cor}{Corollary}
\newtheorem{example}{Example}
\newtheorem{lemma}{Lemma}
\newtheorem{newcommandi}{Definition}


\newcommand{\proof}{\noindent {\bf Proof.\ \ }}

\newcommand{\qed}{\hfill $\square$}


\newcommand{\card}[1]{\vert #1 \vert}

%\newcommand{\qed}{\hspace*{\fill} $\Box$ \bigskip }


%\renewcommand{\thefootnote}{\Alph{footnote}}
\usepackage{fancybox}
\usepackage[french]{babel}
%\usepackage{fullpage}
\usepackage{multicol}
\setlength{\columnseprule}{0.2pt}
\setlength{\columnsep}{16pt}
\usepackage{fancyhdr} % personalisation tete/pied de page
%\pagestyle{fancy}







\usepackage{hyperref}

%\addtolength{\headheight}{50pt}

\setlength{\parindent}{0pt}

\title{Fiche 5.0 : complexité des algorithmes}
\author{BUT Informatique\\
IUT de Vélizy\\
}
\date{}


%\catcode`\_=12 %for escaping underscore

\newcommand{\ww}[1]{\textcolor{white}{#1}}

\newcommand{\code}[1]{%
    \begin{center}
        \tt #1
        \vskip .2cm
        {\tt
            \lstinputlisting[frame=single]{#1}
        }
    \end{center}
}


\usepackage{marginnote}

\usepackage{fancyvrb} % Verbatim avancé

\lstdefinestyle{customc}{
    belowcaptionskip=1\baselineskip,
    breaklines=true,
    frame=single,
    xleftmargin=2cm,
    language=C,
    showstringspaces=false,
    showspaces=false,
    basicstyle=\ttfamily
}


\newcommand{\reflexion}{\hspace{-1.2cm} \includegraphics[width=1cm]{reflexion.jpg} \vskip -.8cm}
%\newcommand{\checkbox}{\includegraphics[width=.5cm]{checkbox.jpg} }
\newcommand{\checkbox}{$\square$ \smallskip}


%%environement pour les icones avec decalage
\newenvironment{icone}[1]{%
    \hskip -.8cm
\begin{tabular}{c|c}
    \hspace{.03\textwidth} \includegraphics[width=.07\textwidth]{#1} & 
\begin{minipage}{.85\textwidth}
}{%
\end{minipage}
\end{tabular}
}





\newcounter{exo} \setcounter{exo}{0}
\newenvironment{action}{%
    \begin{enumerate}[\numerotation] \addtocounter{exo}{-1}%
        }{%
    \end{enumerate}
}

%environement pour liste avec checkbox avec compteur
\newcommand{\numexoa}{\theexo \addtocounter{exo}{1}}
\newcommand{\numerotation}{\checkbox \smallskip \numexoa.}

%%environement de validation
\newenvironment{validation}{%
\smallskip
\begin{tabular}{c|c}
    \hspace{.03\textwidth} \includegraphics[width=.07\textwidth]{teacher.jpg} & 
\begin{minipage}{.85\textwidth}
}{%
\end{minipage}\\
\hline
\end{tabular}
}


%pour les fichiers c et dossiers
\newcounter{exoo} \setcounter{exoo}{0}
\newcommand{\numexo}{\theexoo}
\newcommand{\repexo}{{\tt exo\_\numexo}}
\newcommand{\exoplus}{\addtocounter{exoo}{1}}




\begin{document}
\maketitle





\thispagestyle{empty}

\setcounter{section}{-1}
\section{Tris}
Dans cette partie, vous allez écrire et tester {\em deux tris célèbres et assez simples à mettre en oeuvre}. Puis vous mesurerez leurs performances. Enfin, vous les comparerez au tri natif de python.\ 

Vous travaillerez dans le fichier {\tt tri\_sort.py}, dans lequel vous progresserez au fur et à mesure de la lecture de cet énoncé.

\subsection*{Génération de listes aléatoires}
\begin{action}
\item {\bf Fonction de génération de liste aléatoire} Afin de nourrir vos algorithmes de tri, vous allez écrire une fonction {\tt randlist} qui, pour un entier {\tt n} donné, renvoie une liste de {\tt n} entiers compris entre {\tt 0} et {\tt n-1} et mélangés aléatoirement (faites appel à la fonction {\tt shuffle()}).\ 

Exemple de fonctionnement: {\tt randlist(4)} pourrait renvoyer {\tt [3, 1, 0, 2]}.
\item {\bf Test de {\bf randlist()}} Une fois écrite cette fonction, complétez {\tt test\_randlist()} et vérifiez que {\tt randlist()} fonctionne correctement. {\bf Vous devez ecrire au moins 3 appels de test variés.}
\end{action}

\subsection*{Tri par sélection}
\begin{action}
\item {\bf Découverte du tri par sélection} Le principe du tri par sélection est assez simple:
\begin{enumerate}
  \item Rechercher dans la partie de la liste qui n'est pas encore triée la position de la valeur minimale.
  \item Placer cette valeur en début de tableau en effectuant un échange des deux positions concernées.
  \item Recommencer à l'étape (a) jusqu'à ce que toute la liste soit triée.
\end{enumerate}
Pour vous aider à bien comprendre le principe de ce tri, rendez-vous sur \href{https://visualgo.net/en/sorting}{Visualgo}, sélectionnez dans la barre du haut {\em SEL} (comme {\it selection sort}, le nom anglais du tri par sélection), et lancez le tri en cliquant sur {\em Sort} en bas à gauche.
\item {\bf À vous de sélectionner} Complétez la fonction {\tt tri\_sel()}.
\item {\bf Le moment de vérité} Complétez la fonction {\tt test\_tri\_sel()} qui va vous permettre de tester {\tt tri\_sel()} avec des listes aléatoires générées par {\tt randlist()}.\ 
Effectuez également des tests avec des cas particuliers: {\bf trouvez-en au moins trois.}
\end{action}

\subsection*{Tri à bulles}
\begin{action}
\item {\bf Découverte des bulles} Rendez-vous à nouveau sur \href{https://visualgo.net/en/sorting}{Visualgo} afin de bien comprendre ce que fait le tri à bulles.
\item {\bf À vous de buller} Codez le tri à bulles dans la fonction {\tt tri\_bul()}. Vous penserez à ajouter quelques lignes afin de vérifier s'il est possible d'interrompre l'algorithme quand on détecte que le tableau est trié avant la fin des boucles.
\item {\bf Test des bulles} Complétez la fonction {\tt test\_tri\_bul()} afin de tester votre tri. Cette fonction peut être une simple copie de la fonction {\tt test\_tri\_sel()}.
\end{action}

\section{Mesures de performances}
\subsection{Temps de calcul}
\begin{action}
\item {\bf Chrono reverse} Dans le fichier {\tt cpx\_sort.py}, l'exercice 8 vous montre comment mesurer le temps requis par l'exécution de certaines instructions données. Appelez cette fonction {\tt perf\_reverse()} et observez son fonctionnement.
\item {\bf Perf des tris} Vous allez écrire une fonction {\tt perf\_tris()} qui va mesurer et stocker les performances de vos deux tris pour un nombre variable d'entiers. Par exemple, vous pourriez lancer vos tris sur des tableaux de taille {\tt 100, 200, 300, ..., 2900, 3000}.\  
Vous stockerez  les temps obtenus dans des listes, dans l'ordre d'obtention des résultats.
\end{action}

\subsection{Graphiques}
\begin{action}
\item {\bf Démo pyplot} Dans le fichier {\tt cpx\_sort.py}, l'exercice 10 vous montre comment réaliser un affichage graphique des données d'une liste. Appelez cette fonction {\tt demo\_pyplot()} et observez son fonctionnement.
\item {\bf Plotez} Vous êtes maintenant un as de {\tt pyplot}. Si si. Vous pouvez compléter votre fonction {\tt perf\_tris()} en y ajoutant l'affichage graphique des performances mesurées pour vos deux tris. 
\item {\bf Bravo} Bravo.
\end{action}

\subsection{Comparatif}
\begin{action}
\item {\bf Sa majesté Quick sort} Nous allons maintenant nous mesurer à un adversaire de taille: l'algorithme {\em quick sort} ({\it tri rapide} en français) implémenté dans la méthode {\tt sort()} native des listes de python. Commencez par aller jeter un oeil sur le travail étrange que fait quick sort dans \href{https://visualgo.net/en/sorting}{Visualgo}. 
\item {\bf The race} Dans votre fonction {\tt perf\_tris()}}, ajoutez un appel et une mesure des performances de quicksort. Appelez, mesurez, affichez, pleurez.
\end{action}




\end{document}
